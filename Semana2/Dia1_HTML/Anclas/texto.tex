La salud mental de una persona está profundamente relacionada con
su estabilidad y equilibrio dentro de la sociedad. Esto se ve reflejado
en el desarrollo de capacidades intelectuales, más oportunidades en el
ámbito laboral, y la posibilidad de entablar relaciones interpersonales
con mayor facilidad. Como resultado, el individuo tendrá una mejor
calidad de vida, permitiendo que alcance un estado de bienestar. Sin
embargo, existen una serie de trastornos y enfermedades que pueden
interferir con todo lo antes mencionado.


Entre los problemas más comunes que experimentan millones de 
personas se encuentra la ansiedad. Básicamente, se trata de una 
respuesta involuntaria que es provocada por factores internos o 
externos. Su presencia es bastante común durante la adolescencia, 
pero también forma parte de la vida cotidiana de cualquier 
individuo. Esto se debe a que existen diferentes experiencias que 
pueden ocasionar fuertes respuestas emocionales, las cuales a su vez 
estarán acompañadas de un sentimiento de temor o miedo.

Aunque hay formas de manejar las situaciones antes descritas, en 
algunos casos la ansiedad se convierte en un trastorno. Según muchos
 especialistas en el tema, esto es una consecuencia de la sociedad 
 moderna y sus exigencias, que terminan afectando la salud mental de
  la gente. De este modo, se pueden observar cuadros sintomáticos que 
  están caracterizados por pensamientos negativos y una incapacidad 
  para desenvolverse libremente.


  
Hoy en día hay una mejor comprensión acerca de las consecuencias 
que tienen los trastornos y enfermedades que afectan la salud mental. 
Esto ha permitido que se desarrollen una serie de medicamentos y 
tratamientos para que las personas puedan superar sus síntomas, 
ayudando a que se integren activamente en la sociedad. No obstante, 
en algunos países aún existe mucha incomprensión sobre el tema, solo 
tomándose las medidas necesarias frente a la presencia de desórdenes 
que resultan evidentes. Esto debe cambiar, tanto a nivel social como 
cultural, promoviendo campañas que estén enfocadas a la prevención.
